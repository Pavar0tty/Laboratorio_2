\documentclass[letterpaper,12pt]{article}
\usepackage{mathtools}
\DeclarePairedDelimiter\abs{\lvert}{\rvert}     %serve per mettere il modulo 
\usepackage{booktabs}
\usepackage{bm}
\usepackage{colortbl}
\usepackage{tabularx}
\usepackage{textcomp}
\usepackage{siunitx}
\usepackage{booktabs}
\usepackage{enumitem}
\usepackage{xcolor}
\usepackage{fancyhdr}
\usepackage{caption}
\usepackage{changepage}
\usepackage{amsmath} 
\usepackage{subcaption}
\usepackage{graphicx}
\usepackage[table]{xcolor} 
\usepackage[margin=1in,letterpaper]{geometry} % decreases margins
\usepackage{cite} % takes care of citations
\usepackage[hidelinks]{hyperref} % adds hyper links inside the generated pdf file
% Define the colors
\definecolor{linkcolor}{RGB}{0, 102, 204}
\definecolor{citecolor}{RGB}{34, 139, 34}
\definecolor{urlcolor}{RGB}{255, 69, 0}

% Setup hyperref
\hypersetup{
    colorlinks=false, % colored links
    linkcolor=linkcolor, % color for internal links
    citecolor=citecolor, % color for citations
    urlcolor=urlcolor, % color for URLs
}
\fancypagestyle{logoheader}{
    \fancyhf{}
    \fancyhead[L]{\includegraphics[width = 3cm]{infn-art-science-universita-degli-studi-di-milano-bicocca-maintainer-universita-studi-milano-bicocca.png}}
    \renewcommand{\headrulewidth}{0pt}
    }
\usepackage{blindtext}
\graphicspath{{immagini/}}
%Required for inserting images
%++++++++++++++++++++++++++++++++++++++++
%Margini 



\begin{document}

\title{{\small Università degli studi Milano-Bicocca  Dipartimento di Fisica - Laboratorio II }\\
    Esperienza Ottica - Microonde}
\author{F. Ballo, S. Franceschina, S. Dolci - Gruppo T1 39}
\date{\today}
\maketitle
\thispagestyle{logoheader}


\begin{abstract}
Nella seguente relazione vengono presentati i risultati ottenuti dalla quarta esperienza del corso di Laboratorio II riguardante l'analisi di fenomeni ottici. L'obiettivo di questa esperienza è quello di studiare le proprietà caratteristiche delle onde elettromagnetiche nello spettro delle microonde. Ci si rifà all'utilizzo di emettitori e ricevitori per registrare il segnale delle onde altrimenti invisibili all'occhio umano (lunghezza d'onda circa 2.85cm).  
\begin{adjustwidth}{-1cm}{-1cm}
\end{adjustwidth}
\end{abstract}
\tableofcontents
\newpage

\section{Caratteristiche del fascio}

\subsection{Configurazione del circuito e della strumentazione}
Di seguito riportiamo informazioni sulla strumentazione e sulle modalità di misura

\subsection{Polarizzazione}



\subsection{Ampiezza} 



\subsection{Geometria}


\subsection{Analisi conclusiva fascio}


\section{Angolo di Brewster}

\subsubsection{Analisi dati} 

\section{Interferenza}
Introduzione su interferenza 

\subsection{Specchio Lloyd}
In questa sezione abbiamo utilizzato uno specchio di Lloyd per osservare l'interferenza tra i due fasci di microonde. 
Abbiamo disposto emettitore e ricevitore uno di fronte all'altro, misurandone la distanza $d$, in seguito abbiamo 
posizionato una lastra riflettente ad una certa distanza $h$ dal centro. \\
In questo modo si vengono a creare due fasci: il primo percorre una distanza $d$ in linea retta, mentre il secondo
percorre una distanza $2 \sqrt[]{h^2 + (d/2)^2}$. Tale differenza di percorso porta a delle interferenze: 
se la differenza di cammino ottico è un multiplo intero di $\lambda$ si ha interferenza costruttiva, 
altrimenti si ha interferenza distruttiva.\\
Al fine di misurare la lunghezza d'onda delle microonde, abbiamo seguito le istruzioni fornite dal manuale 
PASCO e abbiamo variato la distanza $h$ alla ricerca di due minimi distanti dieci volte la lunghezza d'onda,
una volta fissata la distanza $d$. Per poter eseguire un confronto sperimentale e non solo con il valore di $\lambda$
tabulato, abbiamo ripetuto la procedura per un'altra distanza $d$. \\
Come formula per il calcolo della lunghezza d'onda abbiamo utilizzato la seguente, ricavata dalle relazioni geometriche
che legano i cammini ottici e la differenza di fase tra i due fasci:
\begin{equation}
    \lambda = \frac{2h + 4 \sqrt[]{(d/2)^2 + h^2}}{n}
    \label{eq:lloyd}
\end{equation}
Dalla prima misurazione abbiamo ottenuto due valori di $\lambda$ che abbiamo mediato, lo stesso abbiamo fatto per 
la seconda misurazione.Come errori delle singole lunghezze d'onda abbiamo propagato gli errori a partire dall'equazione
\ref{eq:lloyd}, in seguito abbiamo propagato gli errori per la media. \\


\subsubsection{Analisi Dati Lloyd}
Riportiamo di seguito i dati raccolti durante l'esperienza e i risultati ottenuti.
Per la prima misurazione abbiamo scelto $d = 100 \pm 1$ cm, mentre per la seconda $d = 110 \pm 1$ cm. Abbiamo stimato
le incertezze di 1 cm poichè sugli "horn" non erano ben segnalati i punti di emissione e di ricezione
dell'onda; non sapendo bene dove fossero localizzati abbiamo aumentato l'errore rispetto alla sensibilità della riga
graduata.\\
In tabella \ref{tab:lloyd1} riportiamo i valori di $h$ e le intensità misurate per i minimi di interferenza, 
in tabella \ref{tab:lloyd2} riportiamo i valori di $h$ e le intensità misurate per i massimi di interferenza.\\
Come valore $\lambda_1$ abbiamo ottenuto $2.86 \pm 0.07$ cm, mentre per $\lambda_2$ abbiamo ottenuto $2.85 \pm 0.06$ cm.\\
Abbiamo confrontato i valori ottenuti con il valore tabulato di $\lambda = 2.85$ cm, ottenendo:
\begin{enumerate}
    \item Distanza in deviazioni standard tra $\lambda_1$ e $\lambda_{tab}$: $0.11 \sigma$
    \item Distanza in deviazioni standard tra $\lambda_2$ e $\lambda_{tab}$: $0.08 \sigma$
    \item Distanza in deviazioni standard tra $\lambda_1$ e $\lambda_2$: $0.03 \sigma$
\end{enumerate}
L'ultimo punto è stato calcolato per verificare la coerenza tra i due valori di $\lambda$ ottenuti.\\

\subsubsection{Specchio Lloyd: Conclusioni}
Dai risultati ottenuti possiamo concludere che la lunghezza d'onda delle microonde calcolata tramite interferometro 
di Lloyd è compatibile con il valore tabulato. 
Al di là della misura della lunghezza d'onda tramite interferenza, questa sezione ci ha permesso di verificare
sperimentalmente un'ipotesi che avanziamo in altre sezioni: l'interferenza per riflessione delle microonde. Ogniqualvolta
un esperimento richieda la rotazione del ricevitore, si possono considerare effetti di interferenza alla Lloyd,
poichè, per angoli in cui ricevitore e emettitore sono sempre più vicini, si può modelizzare il sistema come 
un interferometro di Lloyd.\\
In conclusione, l'interferenza alla Lloyd si ottiene quando un materiale che abbia proprietà riflettenti viene
avvicinato all'apparecchiatura, portandoci a considerare questo effetto di interferenza in tutte le sezioni in cui 
ci troviamo a ruotare il ricevitore.\\


\subsection{Interferometro di Michelson}


\subsubsection{Analisi Dati Michelson}


\section{Diffrazione di Bragg}

\subsection{Analisi Dati Bragg}

\newpage
\section{Tabelle misurazioni}

\begin{table}[h!]
    \centering
    \caption{Lloyd: prima misura}
    \label{tab:lloyd1}
    \begin{tabular}{|c|c|}
        \hline
        $h$ [cm] & $I$ [V]\\
        \hline
        9.9 & 1.64 \\
        16.9 & 1.7 \\
        \hline
    \end{tabular}
\end{table}
    
\begin{table}[h!]
    \centering
    \caption{Lloyd: seconda misura}
    \label{tab:lloyd2}
    \begin{tabular}{|c|c|}
        \hline
        $h$ [cm] & $I$ [V]\\
        \hline
        10.9 & 1.40 \\
        17.2 & 1.5 \\
        \hline
    \end{tabular}
\end{table}

10.9
v4 = 1.40
h4 = 17.2
v4 = 1.5

\end{document}
