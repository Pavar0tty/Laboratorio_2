\documentclass[letterpaper,12pt]{article}
\usepackage{mathtools}
\DeclarePairedDelimiter\abs{\lvert}{\rvert}     %serve per mettere il modulo 
\usepackage{booktabs}
\usepackage{bm}
\usepackage{colortbl}
\usepackage{tabularx}
\usepackage{textcomp}
\usepackage{siunitx}
\usepackage{booktabs}
\usepackage{enumitem}
\usepackage{xcolor}
\usepackage{fancyhdr}
\usepackage{caption}
\usepackage{changepage}
\usepackage{amsmath} 
\usepackage{graphicx}
\usepackage{subcaption}
\usepackage[table]{xcolor} 
\usepackage[margin=1in,letterpaper]{geometry} % decreases margins
\usepackage{cite} % takes care of citations
\usepackage[hidelinks]{hyperref} % adds hyper links inside the generated pdf file
% Define the colors
\definecolor{linkcolor}{RGB}{0, 102, 204}
\definecolor{citecolor}{RGB}{34, 139, 34}
\definecolor{urlcolor}{RGB}{255, 69, 0}

% Setup hyperref
\hypersetup{
    colorlinks=false, % colored links
    linkcolor=linkcolor, % color for internal links
    citecolor=citecolor, % color for citations
    urlcolor=urlcolor, % color for URLs
}
\fancypagestyle{logoheader}{
    \fancyhf{}
    \fancyhead[L]{\includegraphics[width = 3cm]{infn-art-science-universita-degli-studi-di-milano-bicocca-maintainer-universita-studi-milano-bicocca.png}}
    \renewcommand{\headrulewidth}{0pt}
    }
\usepackage{blindtext}
\graphicspath{{immagini/}}
%Required for inserting images
%++++++++++++++++++++++++++++++++++++++++
%Margini 



\begin{document}

\title{{\small Università degli studi Milano-Bicocca  Dipartimento di Fisica - Laboratorio II }\\
    Esperienza Ottica - Microonde}
\author{F. Ballo, S. Franceschina, S. Dolci - Gruppo T1 39}
\date{\today}
\maketitle
\thispagestyle{logoheader}


\begin{abstract}
Nella seguente relazione vengono presentati i risultati ottenuti dalla quarta esperienza del corso di Laboratorio II riguardante l'analisi di fenomeni ottici. L'obiettivo di questa esperienza è quello di studiare le proprietà caratteristiche delle onde elettromagnetiche nello spettro delle microonde. Ci si rifà all'utilizzo di emettitori e ricevitori per registrare il segnale delle onde altrimenti invisibili all'occhio umano (lunghezza d'onda circa 2.85cm).  
\begin{adjustwidth}{-1cm}{-1cm}
\end{adjustwidth}
\end{abstract}
\tableofcontents
\newpage

\section{Caratteristiche del fascio}

\subsection{Configurazione del circuito e della strumentazione}
Di seguito riportiamo informazioni sulla strumentazione e sulle modalità di misura
    

\subsection{Polarizzazione}



\subsection{Ampiezza} 



\subsection{Geometria}


\subsection{Analisi conclusiva fascio}


\section{Angolo di Brewster}

\subsubsection{Analisi dati} 

\section{Interferenza}
Introduzione su interferenza 

\subsection{Specchio Lloyd}

\subsubsection{Analisi Dati Lloyd}

\subsection{Interferometro di Michelson}


\subsubsection{Analisi Dati Michelson}


\section{Diffrazione di Bragg}

\subsection{Analisi Dati Bragg}

\newpage
\section{Tabelle misurazioni}



\end{document}
