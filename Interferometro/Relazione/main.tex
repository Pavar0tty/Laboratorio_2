\documentclass[letterpaper,12pt]{article}
\usepackage{mathtools}
\DeclarePairedDelimiter\abs{\lvert}{\rvert}     %serve per mettere il modulo 
\usepackage{booktabs}
\usepackage{bm}
\usepackage{textcomp}
\usepackage{colortbl}
\usepackage{tabularx}
\usepackage{textcomp}
\usepackage{siunitx}
\usepackage{booktabs}
\usepackage{enumitem}
\usepackage{xcolor}
\usepackage{fancyhdr}
\usepackage{caption}
\usepackage{changepage}
\usepackage{amsmath} 
\usepackage{subcaption}
\usepackage{graphicx}
\usepackage[table]{xcolor} 
\usepackage{colortbl}
\usepackage[margin=1in,letterpaper]{geometry} % decreases margins
\usepackage{cite} % takes care of citations
\usepackage[hidelinks]{hyperref} % adds hyper links inside the generated pdf file
\usepackage{siunitx} % provides the \SI{}{} command for proper typesetting of units
% Define the colors
\definecolor{linkcolor}{RGB}{0, 102, 204}
\definecolor{citecolor}{RGB}{34, 139, 34}
\definecolor{urlcolor}{RGB}{255, 69, 0}
\definecolor{wavelength_406}{RGB}{129, 0, 204}
\definecolor{wavelength_447}{RGB}{0, 53, 255} 
\definecolor{wavelength_402}{RGB}{131, 0, 188}  
\definecolor{wavelength_501}{RGB}{0, 255, 135}
\definecolor{wavelength_440}{RGB}{0, 0, 255}   
\definecolor{wavelength_513}{RGB}{21, 255, 0}   
\definecolor{wavelength_nan}{RGB}{210,210,210} 
\definecolor{wavelength_540}{RGB}{129, 255, 0}  
\definecolor{wavelength_458}{RGB}{0, 113, 255}  
\definecolor{wavelength_568}{RGB}{219, 255, 0}
\definecolor{wavelength_587}{RGB}{255, 233, 0} 
\definecolor{wavelength_585}{RGB}{255, 239, 0} 
\definecolor{wavelength_472}{RGB}{0, 178, 255}
\definecolor{wavelength_667}{RGB}{235, 0, 0}  
\definecolor{wavelength_676}{RGB}{227, 0, 0}  
\definecolor{wavelength_640}{RGB}{255, 33, 0}  
\definecolor{wavelength_696}{RGB}{209, 0, 0}
\definecolor{wavelength_449}{RGB}{0, 65, 255}
\definecolor{wavelength_503}{RGB}{0, 255, 110}
\definecolor{wavelength_581}{RGB}{255, 252, 0}


% Setup hyperref
\hypersetup{
    colorlinks=false, % colored links
    linkcolor=linkcolor, % color for internal links
    citecolor=citecolor, % color for citations
    urlcolor=urlcolor, % color for URLs
}
\fancypagestyle{logoheader}{
    \fancyhf{}
    \fancyhead[L]{\includegraphics[width = 3cm]{infn-art-science-universita-degli-studi-di-milano-bicocca-maintainer-universita-studi-milano-bicocca.png}}
    \renewcommand{\headrulewidth}{0pt}
    }
\usepackage{blindtext}
\graphicspath{{immagini/}}
%Required for inserting images
%++++++++++++++++++++++++++++++++++++++++
%Margini 


\begin{document}


\title{{\small Università degli studi Milano-Bicocca  Dipartimento di Fisica - Laboratorio II }\\
	Esperienza Ottica - Interferometro}
\author{F. Ballo, S. Franceschina, S. Dolci - Gruppo T1 39}
\date{\today}
\maketitle
\thispagestyle{logoheader}


\begin{abstract}
	Nella seguente relazione vengono presentati i risultati ottenuti dalla sesta esperienza del corso di 
    Laboratorio II riguardante l'analisi di fenomeni ottici.
    L'obiettivo di questa esperienza è quello di riprodurre due esperimenti di interferometria: Fabri-Perot e Michelson.
    Per ciascuno di questi setup riprodotti in laboratorio lo scopo è quello di verificare certe relazioni, che occorrono
    nel momento in cui raggi luminosi interferiscono tra loro, dalle quali è possibile ricavare informazioni utili come 
    la lunghezza d'onda della sorgente.
	\begin{adjustwidth}{-1cm}{-1cm}
	\end{adjustwidth}
\end{abstract}
\tableofcontents
\newpage

\section{Configurazione setup esperienza}
Per le misure di questa esperienza abbiamo utilizzato:

\begin{itemize}
    \item Un interferometro di precisione PASCO scientific Modello OS-9255A/OS-9258A , \href{https://www.pasco.com/products/lab-apparatus/light-and-optics/advanced-optics/os-9255}{link.}
    \item Sorgente: laser monocromatico coerente He-Ne con lunghezza d'onda $\lambda = \SI{632.8}{\nano\meter}$.
\end{itemize}


\section{Fabry-Perot}
La prima parte dell'esperienza consiste nella verifica della legge che descrive
i massimi di interferenza, visibili quando due sorgenti si sommano in fase. 
Per farlo abbimao montanto l'interferometro in configurazione Fabry-Perot:

\begin{figure}[ht]
    \centering
    \includegraphics[width=0.4\textwidth]{InterferometroFabry.png}
    \caption{Configurazione Fabry-Perot.}
    \label{fig:fraby-perot config}
\end{figure}

\subsection{Specchio}




\subsection{Frange}



\subsection{Conclusioni Fabry-Perot}



\section{Michelson}

\subsection{Specchio}


\subsection{Frange}


\subsection{Conclusioni Michelson}




\section{Considerazioni sugli errori}
\label{sec:errori}

\subsection{Commenti finali}



\newpage
\section{Tabelle}

\begin{table}[h!]
    \centering
    \begin{tabular}{|c|c|c|c|c|c|c|c|c|}
    \hline
    \multicolumn{2}{|c|}{\textbf{Giallo}} & \multicolumn{2}{|c|}{\textbf{Ciano}} & \multicolumn{2}{|c|}{\textbf{Blu}} & \multicolumn{2}{|c|}{\textbf{Viola}} \\
    \hline
    \textbf{gradi} & \textbf{primi} & \textbf{gradi} & \textbf{primi} & \textbf{gradi} & \textbf{primi} & \textbf{gradi} & \textbf{primi} \\
    \hline
    48 & 5 & 49 & 33 & 50 & 5 & 51 & 1 \\
    48 & 3 & 49 & 36 & 50 & 8 & 51 & 2 \\
    48 & 1 & 49 & 35 & 50 & 8 & 51 & 0 \\
    48 & 0 & 49 & 33 & 50 & 10 & 51 & 1 \\
    48 & 4 & 49 & 34 & 50 & 4 & 51 & 0 \\
    48 & 2 & 49 & 34 & 50 & 5 & 51 & 0 \\
    48 & 2 & 49 & 31 & 50 & 6 & 51 & 1 \\
    48 & 3 & 49 & 34 & 50 & 7 & 51 & 2 \\
    48 & 6 & 49 & 31 & 50 & 5 & 51 & 1 \\
    48 & 2 & 49 & 32 & 50 & 6 & 51 & 0 \\
    \hline
    \end{tabular}
    \caption{Angoli di minima deviazione per mercurio}
    \label{tab:prisma_md}
    \end{table}
\end{document}